\chapter{Contexte général du projet}

\section{Introduction}

Dans ce chapitre, nous allons présenter le cadre général du projet, à savoir :
\begin{itemize}
\item L'organisme d'accueil
\item Le contexte du projet
\item Les objectifs et la démarche de réalisation du projet
\end{itemize}

\section{Présentation de l'organisme d'accueil}

\subsection{DataProtect en quelques mots}

\begin{figure}[H]
\centering
\includegraphics[width=0.5\textwidth]{context/DataProtect_is_the_company_responsible_for_delivering_the_training_program_providing_the_technical_instruction_and_practical_learning_content.png}
\caption{Logo de l'entreprise DataProtect}
\label{fig:dataprotect-logo}
\end{figure}

DATAPROTECT est une entreprise spécialisée en sécurité de l'information. Fondée par Ali EL AZZOUZI, un expert en sécurité de l'information ayant mené plusieurs projets de conseil et d'intégration de solutions de sécurité au Maroc et à l'étranger, DATAPROTECT appuie son offre sur une vision unifiée de la sécurité de l'information. Dotée d'un réservoir de compétences pointues en sécurité lui permettant d'assurer une expertise unique sur le marché local et régional.

Depuis sa création, DATAPROTECT ne cesse d'évoluer pour délivrer ses prestations d'excellence à travers une équipe d'experts pluridisciplinaires dotée d'un sens unique de l'intimité client. Aussi, son statut de première entité accréditée PCI QSA au Maroc par le consortium Payment Card Industry Security Standards Council pour les certifications PCI DSS et PA DSS, fait d'elle un cas d'école unique dans la région.

Avec une centaine de clients en Afrique du Nord, en Afrique Subsaharienne et au Moyen Orient, DATAPROTECT est aujourd'hui capable de délivrer ses services en toute agilité, pour des multinationales comme pour des entreprises locales, avec à la clé une réputation établie de pionnier sur la thématique de la sécurité de l'Information.

\subsection{DataProtect : Un acteur clé dans le marché de la cybersécurité}

Créée en 2009, DATAPROTECT, aujourd'hui c'est :
\begin{itemize}
\item +150 employés dont 110 consultants « Full Security »
\item +200 certifications sécurité
\item +350 clients actifs dont une centaine de banques
\item +1500 projets à travers 35 pays
\item +16 M Euro de chiffre d'affaires dont 80\% à l'international (chiffre 2020)
\item Une filiale en France depuis 2016
\item Une équipe dédiée au développement des solutions innovantes en Cybersécurité
\end{itemize}

\subsection{Jobintech : Organisme de financement}

\begin{figure}[H]
\centering
\includegraphics[width=0.6\textwidth]{context/Jobintech_is_the_organization_that_fully_funds_the_training_program_enabling_participants_to_attend_the_formation_free_of_charge.png}
\caption{Jobintech - Organisme de financement de la formation}
\label{fig:jobintech}
\end{figure}

Jobintech est l'organisation qui finance intégralement le programme de formation, permettant aux participants d'assister à la formation gratuitement. Cette collaboration entre DataProtect (fournisseur de formation) et Jobintech (organisme de financement) permet de former des professionnels de la cybersécurité et de contribuer au développement des compétences dans le domaine de la sécurité informatique.

\subsection{Activités de DataProtect}

DATAPROTECT est organisée autour de cinq pôles d'activités :

\subsubsection{Cyberdefense - Intégration}

DATAPROTECT offre l'intégration de solutions pour la sécurisation du poste de travail qui se matérialise par :
\begin{itemize}
\item La mise en place de l'antivirus
\item La mise en place de solution de gestion de vulnérabilités
\item La mise en place de solution de gestion des correctifs
\item La mise en place de solution de prévention de fuite d'informations sensibles
\item La mise en place d'outils de cryptage de données
\item Définition des programmes de formation adaptés à chaque profil du personnel exploitant identifié
\end{itemize}

\subsubsection{Gouvernance, Risk \& Compliance Conseil}

Ayant mené une centaine de missions d'audit de sécurité des systèmes d'information pour le compte d'organisations exerçants dans divers domaines d'activité, DATAPROTECT dispose aujourd'hui d'un retour d'expérience très riche et varié en la matière. Des tests d'intrusion externes jusqu'à l'audit de code applicatif, en passant par les tests d'intrusion internes, l'audit des configurations, l'audit d'architecture de sécurité, l'audit de sécurité de poste de travail et l'audit organisationnel de sécurité, l'équipe DATAPROTECT est en mesure d'évaluer les différentes dimensions de la sécurité de l'information.

\subsubsection{MSSP - Infogérance}

À la demande de plusieurs de ses clients, DATAPROTECT a mis en place un « Security Operations Center » (SOC) dont la fonction principale est de fournir des services de détection et de traitement des incidents de sécurité. Le centre de sécurité collecte ainsi les événements (sous forme de logs notamment) remontés par les composants de sécurité, les analyse, détecte les anomalies et définit des réactions en cas d'émission d'alerte.

\subsubsection{Formation}

Les formations proposées sont particulièrement adaptées aux besoins du marché. Elles touchent tous les domaines liés à la sécurité du système d'information : sécurité de l'information, management des risques IT, services IT, sécurité applicative, continuité d'activité, préparation aux certifications, sensibilisation et audit, protection des données à caractère personnel, ateliers pratiques.

\subsubsection{Security Intelligence (SOC)}

DATAPROTECT a ouvert le premier SOC au Maroc en 2014 afin de répondre à une vague d'incidents parmi les opérateurs d'importance vitale. Aujourd'hui le SOC compte 24 collaborateurs à plein temps. Les principales activités du SOC DATAPROTECT sont :
\begin{itemize}
\item Surveillance des incidents sécurité 24/7
\item Assistance à la réponse aux incidents
\item Ingénierie SIEM
\item Infogérance des solutions de sécurité
\item Maintenance préventive et curative
\item Investigation et Compromise Assessment
\item Veille et publication des bulletins de sécurité
\item Analyse des malwares
\end{itemize}

\section{Présentation du projet}

\subsection{Étude de l'existant}

Une étape essentielle de tout projet consiste à effectuer une étude préalable. Cette étude consiste à examiner la problématique que nous allons attaquer afin de déclarer les défaillances et les insuffisances du système. En effet, dans le cas général la mise en place d'un projet est due à un problème ou un manque. Il faut donc bien étudier l'existant pour obtenir des résultats efficaces.

\subsection{Problématique}

Bien que les centres opérationnels de sécurité (SOC) s'appuient largement sur les outils SIEM pour surveiller les environnements Windows, la détection réelle des attaques avancées reste un défi majeur. Les attaquants utilisent aujourd'hui des techniques de plus en plus furtives, basées notamment sur PowerShell, Sysmon bypassing, living-off-the-land, ou encore des mécanismes d'exfiltration discrets difficiles à identifier dans les journaux traditionnels.

Dans ce contexte, les organisations se trouvent confrontées à une problématique centrale :

\textit{Comment détecter efficacement une attaque lorsqu'elle est menée de manière silencieuse, progressive et en utilisant des outils légitimes du système Windows ?}

De plus, malgré la richesse des logs Windows (Event Logs, Sysmon, PowerShell), la corrélation entre ces données reste complexe, et les analystes peinent souvent à reconstruire le chemin d'attaque complet à partir des seuls journaux collectés par le SIEM.

La difficulté s'accentue face à :
\begin{itemize}
\item La volumétrie importante des événements collectés
\item La variabilité des formats de logs
\item L'absence de cas d'usage SIEM suffisamment précis
\item L'évolution constante des techniques d'attaque documentées dans MITRE ATT\&CK
\end{itemize}

Ainsi, il devient essentiel d'adopter une démarche structurée de Threat Hunting, permettant :
\begin{itemize}
\item De formuler des hypothèses d'investigation
\item De rechercher activement les comportements suspects
\item De corréler les actions observées dans les logs
\item De reconstituer une timeline d'attaque exploitable
\end{itemize}

\subsection{Gestion du projet}

\subsubsection{Planification du projet}

La gestion de projet joue un rôle essentiel dans la réussite de tout travail technique. Elle permet d'assurer un bon déroulement des différentes phases, en définissant à l'avance les étapes, les délais, et les livrables associés. Dans le cadre de ce projet, nous avons adopté une planification structurée, permettant de visualiser clairement l'enchaînement des tâches et de gérer efficacement le temps imparti.

Le projet a été découpé en plusieurs phases clés : étude du besoin, analyse, conception de l'environnement, mise en œuvre technique, tests, validation et rédaction du rapport. Pour chaque tâche, nous avons défini une durée, des dates de début et de fin, ainsi que les dépendances entre les étapes.

\subsection{Objectifs du stage}

Dans un contexte où les cyberattaques deviennent de plus en plus sophistiquées, DataProtect vise à renforcer la détection des incidents sur les environnements Windows. Ce projet consiste à simuler et analyser une attaque de type AS-REP Roasting suivie d'exfiltration de données, en exploitant uniquement les logs collectés par le SIEM (Event Logs, Sysmon, PowerShell, réseau).

L'objectif est de :
\begin{itemize}
\item Reconstituer la timeline de l'attaque
\item Identifier les comportements suspects
\item Proposer des règles SIEM efficaces
\item Améliorer la réactivité et la couverture de détection des analystes SOC
\end{itemize}

\section{Conclusion}

À l'issue de ce premier chapitre, notre projet de Threat Hunting sur un environnement Windows est clairement ancré dans son contexte. Nous avons présenté l'organisation d'accueil, DataProtect, et défini les enjeux liés à la détection des attaques sophistiquées. Nous avons également exposé la démarche adoptée pour simuler l'attaque, analyser les logs via le SIEM et reconstruire la timeline des événements, afin de renforcer la visibilité et la réactivité des analystes SOC. Les chapitres suivants détailleront les besoins, les objectifs et les étapes techniques pour atteindre une couverture optimale de détection et proposer des cas d'usage SIEM adaptés.
