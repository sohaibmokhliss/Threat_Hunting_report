\chapter{Résultats et recommandations}

\section{Introduction}

Ce chapitre présente les résultats de l'investigation menée, l'impact de l'incident détecté, et propose des recommandations pour améliorer la posture de sécurité et les capacités de détection du SOC.

\section{Synthèse des résultats}

\subsection{Incident confirmé}

L'investigation a confirmé une compromission réelle du serveur Windows par un acteur externe utilisant des techniques sophistiquées. L'attaque a suivi une chaîne d'attaque complète :

\begin{enumerate}
\item \textbf{Credential Access :} AS-REP Roasting pour compromettre un compte utilisateur
\item \textbf{Initial Access :} Authentification RDP avec credentials valides
\item \textbf{Execution :} Commandes PowerShell encodées pour éviter la détection
\item \textbf{Discovery :} Reconnaissance du système et énumération des fichiers sensibles
\item \textbf{Persistence :} Création d'une tâche planifiée malveillante
\item \textbf{Exfiltration :} Transfert de données via BITS
\end{enumerate}

\subsection{Efficacité de la méthode ACH}

La méthode ACH s'est révélée particulièrement efficace pour :
\begin{itemize}
\item Structurer l'investigation de manière rigoureuse
\item Éviter les biais de confirmation en considérant plusieurs hypothèses
\item Documenter le raisonnement analytique de manière traçable
\item Fournir une conclusion avec un niveau de confiance quantifiable
\end{itemize}

\subsection{Couverture de détection}

L'analyse a permis d'évaluer la couverture de détection actuelle :

\begin{table}[H]
\centering
\begin{tabular}{@{}llcc@{}}
\toprule
\textbf{Phase} & \textbf{Technique} & \textbf{Détectable} & \textbf{Alerté} \\ \midrule
Credential Access & AS-REP Roasting & ✔️ & ❌ \\
Initial Access & RDP Logon & ✔️ & ✔️ \\
Execution & PowerShell Encoded & ✔️ & ⚠️ \\
Discovery & File Enumeration & ✔️ & ❌ \\
Persistence & Scheduled Task & ✔️ & ❌ \\
Exfiltration & BITS Transfer & ✔️ & ❌ \\ \bottomrule
\end{tabular}
\caption{Évaluation de la couverture de détection}
\label{tab:detection-coverage}
\end{table}

\textit{Légende : ✔️ = Détectable dans les logs, ⚠️ = Partiellement alerté, ❌ = Non alerté}

\section{Évaluation de l'impact}

\subsection{Impact technique}

\begin{description}
\item[Confidentialité] \textbf{ÉLEVÉE} - Des fichiers sensibles (backups, potentiellement des credentials) ont été exfiltrés
\item[Intégrité] \textbf{MOYENNE} - Le système a été modifié (tâche planifiée, scripts)
\item[Disponibilité] \textbf{FAIBLE} - Aucune dégradation de service détectée
\end{description}

\subsection{Impact métier}

\begin{itemize}
\item \textbf{Exposition de données sensibles} : Risque de fuite d'informations confidentielles
\item \textbf{Perte de confiance} : Impact sur la réputation si l'incident est public
\item \textbf{Coûts de remédiation} : Temps et ressources nécessaires pour l'éradication
\item \textbf{Conformité réglementaire} : Obligation de notification selon RGPD si données personnelles affectées
\end{itemize}

\subsection{Dommages potentiels évités}

Grâce à la détection et à l'investigation, plusieurs scénarios aggravants ont été évités :
\begin{itemize}
\item Mouvement latéral vers d'autres systèmes
\item Élévation de privilèges vers Domain Admin
\item Déploiement de ransomware
\item Exfiltration massive de données
\end{itemize}

\section{Recommandations stratégiques}

\subsection{Amélioration de la posture de sécurité}

\subsubsection{Protection contre l'AS-REP Roasting}

\textbf{Action immédiate :}
\begin{lstlisting}[language=powershell, caption={Script pour identifier les comptes vulnérables}]
# Identifier tous les comptes sans pré-authentification
Get-ADUser -Filter {DoesNotRequirePreAuth -eq $true} | 
Select-Object Name, SamAccountName, DoesNotRequirePreAuth
\end{lstlisting}

\textbf{Recommandations :}
\begin{itemize}
\item Activer la pré-authentification Kerberos pour tous les comptes
\item Si un compte doit garder cette configuration, renforcer le mot de passe (25+ caractères aléatoires)
\item Auditer régulièrement les configurations AD
\end{itemize}

\subsubsection{Durcissement Active Directory}

\begin{enumerate}
\item \textbf{Politique de mots de passe renforcée}
   \begin{itemize}
   \item Longueur minimale : 14 caractères
   \item Complexité obligatoire
   \item Expiration : 90 jours maximum
   \item Historique : 24 mots de passe
   \end{itemize}

\item \textbf{Authentification multifacteur (MFA)}
   \begin{itemize}
   \item Déployer MFA pour tous les accès RDP
   \item Privilégier l'authentification par certificat
   \item Utiliser des solutions comme Azure MFA ou RADIUS
   \end{itemize}

\item \textbf{Principe du moindre privilège}
   \begin{itemize}
   \item Limiter les comptes avec privilèges Domain Admin
   \item Utiliser des comptes dédiés pour l'administration
   \item Implémenter des PAW (Privileged Access Workstations)
   \end{itemize}
\end{enumerate}

\subsubsection{Sécurisation PowerShell}

\begin{enumerate}
\item \textbf{Activer Constrained Language Mode}
\begin{lstlisting}[language=powershell, caption={Configuration du Constrained Language Mode}]
# Via GPO : Set Environment Variable
$env:__PSLockdownPolicy = 4
\end{lstlisting}

\item \textbf{Application Whitelisting}
   \begin{itemize}
   \item Déployer AppLocker ou Windows Defender Application Control
   \item Restreindre l'exécution de PowerShell aux administrateurs
   \end{itemize}

\item \textbf{Enhanced Logging}
   \begin{itemize}
   \item Module Logging activé
   \item Script Block Logging activé
   \item Transcription PowerShell activée
   \end{itemize}
\end{enumerate}

\subsection{Amélioration des capacités de détection}

\subsubsection{Nouveaux cas d'usage SIEM}

\textbf{Use Case 1 : Détection AS-REP Roasting}

\begin{lstlisting}[language=SQL, caption={Règle de détection AS-REP Roasting}]
rule_name: "AS-REP Roasting Activity Detected"
query: |
  event.code: 4768 AND 
  winlog.event_data.PreAuthType: 0
severity: HIGH
description: "Kerberos ticket requested without pre-authentication"
recommendation: "Investigate account and enable pre-authentication"
\end{lstlisting}

\textbf{Use Case 2 : PowerShell encodé suspect}

\begin{lstlisting}[language=SQL, caption={Règle de détection PowerShell encodé}]
rule_name: "Suspicious Encoded PowerShell Execution"
query: |
  event.code: 4104 AND 
  (message: "*-Enc*" OR message: "*-EncodedCommand*") AND
  (message: "*Get-ChildItem*" OR message: "*Invoke-*")
severity: MEDIUM
description: "Encoded PowerShell with suspicious cmdlets"
\end{lstlisting}

\textbf{Use Case 3 : Tâche planifiée malveillante}

\begin{lstlisting}[language=SQL, caption={Règle de détection tâches suspectes}]
rule_name: "Suspicious Scheduled Task Creation"
query: |
  event.code: 4698 AND 
  (process.command_line: "*powershell*" OR 
   process.command_line: "*-Enc*" OR
   process.command_line: "*bitsadmin*")
severity: HIGH
description: "Scheduled task created with suspicious command"
\end{lstlisting}

\textbf{Use Case 4 : Exfiltration via BITS}

\begin{lstlisting}[language=SQL, caption={Règle de détection exfiltration BITS}]
rule_name: "Data Exfiltration via BITS"
query: |
  process.name: "bitsadmin.exe" AND 
  (process.command_line: "*upload*" OR 
   process.command_line: "*addfile*") AND
  NOT destination.ip: (192.168.0.0/16 OR 10.0.0.0/8)
severity: CRITICAL
description: "BITS used for potential data exfiltration"
\end{lstlisting}

\textbf{Use Case 5 : Authentification anormale}

\begin{lstlisting}[language=SQL, caption={Règle de détection logon anormal}]
rule_name: "Abnormal Interactive Logon to DC"
query: |
  event.code: 4624 AND 
  winlog.event_data.LogonType: 10 AND
  host.hostname: "DC-*" AND
  hour_of_day: (0-7 OR 20-23)
severity: MEDIUM
description: "Interactive logon to DC outside business hours"
\end{lstlisting}

\subsubsection{Corrélation avancée}

Implémenter des règles de corrélation pour détecter la chaîne d'attaque complète :

\begin{lstlisting}[caption={Règle de corrélation - Chaîne d'attaque}]
rule_name: "Suspicious Activity Chain Detected"
correlation: |
  SEQUENCE by user.name
    [Event 4768 with PreAuthType=0]
    [Event 4624 with LogonType=10] within 30m
    [Event 4104 with encoded PowerShell] within 10m
    [Event 4698 scheduled task creation] within 20m
severity: CRITICAL
description: "Full attack chain detected - immediate investigation required"
\end{lstlisting}

\subsection{Amélioration des processus SOC}

\subsubsection{Threat Hunting proactif}

\begin{enumerate}
\item \textbf{Hunting régulier}
   \begin{itemize}
   \item Sessions de hunting hebdomadaires dédiées
   \item Rotation des analysts sur différentes hypothèses
   \item Documentation des techniques recherchées
   \end{itemize}

\item \textbf{Utilisation systématique de l'ACH}
   \begin{itemize}
   \item Former tous les analystes à la méthode ACH
   \item Template standardisé pour les investigations
   \item Revue par les pairs des analyses ACH
   \end{itemize}

\item \textbf{Threat Intelligence}
   \begin{itemize}
   \item Abonnement à des flux de threat intelligence
   \item Intégration avec MITRE ATT\&CK
   \item Veille sur les nouvelles TTPs
   \end{itemize}
\end{enumerate}

\subsubsection{Formation et sensibilisation}

\begin{itemize}
\item \textbf{Formation technique :} PowerShell avancé, Kerberos, Active Directory
\item \textbf{Formation méthodologique :} ACH, Threat Hunting, incident response
\item \textbf{Exercices pratiques :} Simulations d'attaque régulières (Purple Team)
\item \textbf{Sensibilisation :} Sessions de sensibilisation pour les utilisateurs finaux
\end{itemize}

\section{Métriques de performance}

\subsection{KPIs proposés}

\begin{table}[H]
\centering
\begin{tabular}{@{}lll@{}}
\toprule
\textbf{Métrique} & \textbf{Objectif} & \textbf{Fréquence} \\ \midrule
Mean Time to Detect (MTTD) & < 2 heures & Mensuel \\
Mean Time to Respond (MTTR) & < 4 heures & Mensuel \\
False Positive Rate & < 10\% & Hebdomadaire \\
Alert Coverage & > 90\% techniques ATT\&CK & Trimestriel \\
Hunting Sessions & > 4 par mois & Mensuel \\
Incidents détectés proactivement & > 30\% & Trimestriel \\ \bottomrule
\end{tabular}
\caption{KPIs proposés pour le SOC}
\label{tab:kpis}
\end{table}

\subsection{Tableau de bord de suivi}

Mise en place d'un tableau de bord Kibana pour suivre :
\begin{itemize}
\item Volume d'événements collectés par source
\item Alertes générées par sévérité
\item Temps de réponse aux incidents
\item Couverture MITRE ATT\&CK
\item Résultats des sessions de hunting
\end{itemize}

\section{Plan d'amélioration continue}

\subsection{Court terme (1-3 mois)}

\begin{enumerate}
\item Corriger la vulnérabilité AS-REP Roasting
\item Déployer les 5 nouveaux cas d'usage SIEM
\item Former les analystes à la méthode ACH
\item Activer MFA pour RDP
\end{enumerate}

\subsection{Moyen terme (3-6 mois)}

\begin{enumerate}
\item Implémenter AppLocker / WDAC
\item Déployer un programme de Threat Hunting structuré
\item Intégrer des flux de Threat Intelligence
\item Automatiser certaines réponses (SOAR)
\end{enumerate}

\subsection{Long terme (6-12 mois)}

\begin{enumerate}
\item Déployer une solution EDR
\item Implémenter un programme Purple Team
\item Obtenir la certification SOC pour l'équipe
\item Développer des playbooks automatisés avancés
\end{enumerate}

\section{Conclusion}

Ce chapitre a présenté les résultats de l'investigation et proposé des recommandations concrètes pour améliorer la détection et la prévention des attaques similaires. Les recommandations couvrent trois axes principaux :

\begin{enumerate}
\item \textbf{Durcissement de la sécurité :} Corrections des vulnérabilités identifiées
\item \textbf{Amélioration de la détection :} Nouveaux cas d'usage SIEM et corrélation avancée
\item \textbf{Optimisation des processus :} Threat Hunting proactif et méthode ACH
\end{enumerate}

La mise en œuvre de ces recommandations permettra de renforcer significativement la posture de sécurité et les capacités opérationnelles du SOC face aux menaces avancées.
